\documentclass{report}

\usepackage{blindtext}
\usepackage[most]{tcolorbox}
\usepackage{graphicx}
\graphicspath{ {./Images/} }

\makeatletter
\NewDocumentCommand{\mynote}{+O{}+m}{%
  \begingroup
  \tcbset{%
    noteshift/.store in=\mynote@shift,
    noteshift=1.5cm
  }
  \begin{tcolorbox}[nobeforeafter,
    enhanced,
    sharp corners,
    toprule=1pt,
    bottomrule=1pt,
    leftrule=0pt,
    rightrule=0pt,
    colback=yellow!20,
    #1,
    left skip=\mynote@shift,
    right skip=\mynote@shift,
    overlay={\node[right] (mynotenode) at ([xshift=-\mynote@shift]frame.west) {\textbf{Note:}} ;},
    ]
    #2
  \end{tcolorbox}
  \endgroup
  }
\makeatother

\begin{document}

\title{Imterim Report}
\author{267533}

\maketitle

\section{Introduction}
While the written word can gain a form of dynamism through ambiguity, applying that to an interactive and visual medium becomes challenging to do concretely. Instead, gameplay elements, such as simulation and customisation, can be used to engender a narrative in the player, giving them the space to fill in the gaps in their own minds. When done well even non impactful decisions can be imbued with a sense of weight, this feeling that everything contributes. On the other end when done poorly, all decisions can become meaningless whether they are impactful or not. What matters is the player believing they're having an impact. 

This form of narrative interaction is seen commonly in the large-scale real-time strategy genre (e.g. Stellaris, Crusader Kings, etc.) where interlocking systems tell the stories of large factions. The objective of this project is to first create a beta version of a game in this genre, then perform a round of user testing. Insights gained from that process will then be used to complete a vertical slice. Given the constraints inherently imposed by solo development across such a limited time scale, it is unlikely a fully finalized version of the game will be completed.

To allow the player to make immediate, impactful decisions at the beginning of any given run, unlike most games in the genre, the player's power (i.e. their ability to affect the simulated environment) will be organized differently to the other factions. Instead of playing as a larger force the player will be a single powerful entity (i.e. a single ship, a single character, etc.) with a large amount of upfront ability to effect change. Additionally, they will be an outsider to the game world, allowing us to include a wider possibility space of "point of origins" for the player's character, incentivising players to immediately begin thinking about where they came from. Where a character comes from informs where they are going. These early decisions should give players an idea of who the character they are going to be playing is, and impact their future roleplay decisions.
\newline
\newline
This report consists of several sections, the first of which is Professional Considerations, where user testing and acquiring consent for personal data collection is discussed in detail. Next is Requirements Analysis, which outlines the objectives of the project in distinct steps alongside some additional extension goals (e.g. additonal user testing, adding additional simulation elements). The section after that is the full Project Plan, consisting of a breakdown of the work required to complete the project into distinct phases. Finally, there is an Interim Log detailing the steps of the Project Plan completed as of the submission of this Interim Report.
\end{document}